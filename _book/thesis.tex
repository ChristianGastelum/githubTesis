% This is the Reed College LaTeX thesis template. Most of the work
% for the document class was done by Sam Noble (SN), as well as this
% template. Later comments etc. by Ben Salzberg (BTS). Additional
% restructuring and APA support by Jess Youngberg (JY).
% Your comments and suggestions are more than welcome; please email
% them to cus@reed.edu
%
% See http://web.reed.edu/cis/help/latex.html for help. There are a
% great bunch of help pages there, with notes on
% getting started, bibtex, etc. Go there and read it if you're not
% already familiar with LaTeX.
%
% Any line that starts with a percent symbol is a comment.
% They won't show up in the document, and are useful for notes
% to yourself and explaining commands.
% Commenting also removes a line from the document;
% very handy for troubleshooting problems. -BTS

% As far as I know, this follows the requirements laid out in
% the 2002-2003 Senior Handbook. Ask a librarian to check the
% document before binding. -SN

%%
%% Preamble
%%
% \documentclass{<something>} must begin each LaTeX document
\documentclass[12pt,twoside]{reedthesis}
% Packages are extensions to the basic LaTeX functions. Whatever you
% want to typeset, there is probably a package out there for it.
% Chemistry (chemtex), screenplays, you name it.
% Check out CTAN to see: http://www.ctan.org/
%%
\usepackage{graphicx,latexsym}
\usepackage{amsmath}
\usepackage{amssymb,amsthm}
\usepackage{longtable,booktabs,setspace}
\usepackage{chemarr} %% Useful for one reaction arrow, useless if you're not a chem major
\usepackage[hyphens]{url}
% Added by CII
\usepackage{hyperref}
\usepackage{lmodern}
\usepackage{float}
\floatplacement{figure}{H}
% End of CII addition
\usepackage{rotating}

\usepackage[utf8]{inputenc}
\usepackage[spanish]{babel}
% Next line commented out by CII
%%% \usepackage{natbib}
% Comment out the natbib line above and uncomment the following two lines to use the new
% biblatex-chicago style, for Chicago A. Also make some changes at the end where the
% bibliography is included.
%\usepackage{biblatex-chicago}
%\bibliography{thesis}


% Added by CII (Thanks, Hadley!)
% Use ref for internal links
\renewcommand{\hyperref}[2][???]{\autoref{#1}}
\def\chapterautorefname{Chapter}
\def\sectionautorefname{Section}
\def\subsectionautorefname{Subsection}
% End of CII addition

% Added by CII
\usepackage{caption}
\captionsetup{width=5in}
% End of CII addition

% \usepackage{times} % other fonts are available like times, bookman, charter, palatino

% Syntax highlighting #22
  \usepackage{color}
  \usepackage{fancyvrb}
  \newcommand{\VerbBar}{|}
  \newcommand{\VERB}{\Verb[commandchars=\\\{\}]}
  \DefineVerbatimEnvironment{Highlighting}{Verbatim}{commandchars=\\\{\}}
  % Add ',fontsize=\small' for more characters per line
  \usepackage{framed}
  \definecolor{shadecolor}{RGB}{248,248,248}
  \newenvironment{Shaded}{\begin{snugshade}}{\end{snugshade}}
  \newcommand{\AlertTok}[1]{\textcolor[rgb]{0.94,0.16,0.16}{#1}}
  \newcommand{\AnnotationTok}[1]{\textcolor[rgb]{0.56,0.35,0.01}{\textbf{\textit{#1}}}}
  \newcommand{\AttributeTok}[1]{\textcolor[rgb]{0.77,0.63,0.00}{#1}}
  \newcommand{\BaseNTok}[1]{\textcolor[rgb]{0.00,0.00,0.81}{#1}}
  \newcommand{\BuiltInTok}[1]{#1}
  \newcommand{\CharTok}[1]{\textcolor[rgb]{0.31,0.60,0.02}{#1}}
  \newcommand{\CommentTok}[1]{\textcolor[rgb]{0.56,0.35,0.01}{\textit{#1}}}
  \newcommand{\CommentVarTok}[1]{\textcolor[rgb]{0.56,0.35,0.01}{\textbf{\textit{#1}}}}
  \newcommand{\ConstantTok}[1]{\textcolor[rgb]{0.00,0.00,0.00}{#1}}
  \newcommand{\ControlFlowTok}[1]{\textcolor[rgb]{0.13,0.29,0.53}{\textbf{#1}}}
  \newcommand{\DataTypeTok}[1]{\textcolor[rgb]{0.13,0.29,0.53}{#1}}
  \newcommand{\DecValTok}[1]{\textcolor[rgb]{0.00,0.00,0.81}{#1}}
  \newcommand{\DocumentationTok}[1]{\textcolor[rgb]{0.56,0.35,0.01}{\textbf{\textit{#1}}}}
  \newcommand{\ErrorTok}[1]{\textcolor[rgb]{0.64,0.00,0.00}{\textbf{#1}}}
  \newcommand{\ExtensionTok}[1]{#1}
  \newcommand{\FloatTok}[1]{\textcolor[rgb]{0.00,0.00,0.81}{#1}}
  \newcommand{\FunctionTok}[1]{\textcolor[rgb]{0.00,0.00,0.00}{#1}}
  \newcommand{\ImportTok}[1]{#1}
  \newcommand{\InformationTok}[1]{\textcolor[rgb]{0.56,0.35,0.01}{\textbf{\textit{#1}}}}
  \newcommand{\KeywordTok}[1]{\textcolor[rgb]{0.13,0.29,0.53}{\textbf{#1}}}
  \newcommand{\NormalTok}[1]{#1}
  \newcommand{\OperatorTok}[1]{\textcolor[rgb]{0.81,0.36,0.00}{\textbf{#1}}}
  \newcommand{\OtherTok}[1]{\textcolor[rgb]{0.56,0.35,0.01}{#1}}
  \newcommand{\PreprocessorTok}[1]{\textcolor[rgb]{0.56,0.35,0.01}{\textit{#1}}}
  \newcommand{\RegionMarkerTok}[1]{#1}
  \newcommand{\SpecialCharTok}[1]{\textcolor[rgb]{0.00,0.00,0.00}{#1}}
  \newcommand{\SpecialStringTok}[1]{\textcolor[rgb]{0.31,0.60,0.02}{#1}}
  \newcommand{\StringTok}[1]{\textcolor[rgb]{0.31,0.60,0.02}{#1}}
  \newcommand{\VariableTok}[1]{\textcolor[rgb]{0.00,0.00,0.00}{#1}}
  \newcommand{\VerbatimStringTok}[1]{\textcolor[rgb]{0.31,0.60,0.02}{#1}}
  \newcommand{\WarningTok}[1]{\textcolor[rgb]{0.56,0.35,0.01}{\textbf{\textit{#1}}}}

% To pass between YAML and LaTeX the dollar signs are added by CII
\title{My Final College Paper}
\author{Your R. Name}
% The month and year that you submit your FINAL draft TO THE LIBRARY (May or December)
\date{May 20xx}
\division{Mathematics and Natural Sciences}
\advisor{Advisor F. Name}
\institution{Reed College}
\degree{Bachelor of Arts}
%If you have two advisors for some reason, you can use the following
% Uncommented out by CII
% End of CII addition

%%% Remember to use the correct department!
\department{Mathematics}
% if you're writing a thesis in an interdisciplinary major,
% uncomment the line below and change the text as appropriate.
% check the Senior Handbook if unsure.
%\thedivisionof{The Established Interdisciplinary Committee for}
% if you want the approval page to say "Approved for the Committee",
% uncomment the next line
%\approvedforthe{Committee}

% Added by CII
%%% Copied from knitr
%% maxwidth is the original width if it's less than linewidth
%% otherwise use linewidth (to make sure the graphics do not exceed the margin)
\makeatletter
\def\maxwidth{ %
  \ifdim\Gin@nat@width>\linewidth
    \linewidth
  \else
    \Gin@nat@width
  \fi
}
\makeatother

\renewcommand{\contentsname}{Table of Contents}
% End of CII addition

\setlength{\parskip}{0pt}

% Added by CII
  %\setlength{\parskip}{\baselineskip}
  \usepackage[parfill]{parskip}

\providecommand{\tightlist}{%
  \setlength{\itemsep}{0pt}\setlength{\parskip}{0pt}}

\Acknowledgements{
I want to thank a few people.
}

\Dedication{
You can have a dedication here if you wish.
}

\Preface{
This is an example of a thesis setup to use the reed thesis document
class (for LaTeX) and the R bookdown package, in general.
}

\Abstract{
The preface pretty much says it all.

\par

Second paragraph of abstract starts here.
}

% End of CII addition
%%
%% End Preamble
%%
%

\usepackage{amsthm}
\newtheorem{theorem}{Theorem}[chapter]
\newtheorem{lemma}{Lemma}[chapter]
\newtheorem{corollary}{Corollary}[chapter]
\newtheorem{proposition}{Proposition}[chapter]
\newtheorem{conjecture}{Conjecture}[chapter]
\theoremstyle{definition}
\newtheorem{definition}{Definition}[chapter]
\theoremstyle{definition}
\newtheorem{example}{Example}[chapter]
\theoremstyle{definition}
\newtheorem{exercise}{Exercise}[chapter]
\theoremstyle{remark}
\newtheorem*{remark}{Remark}
\newtheorem*{solution}{Solution}
\begin{document}

% Everything below added by CII
  \maketitle

\frontmatter % this stuff will be roman-numbered
\pagestyle{empty} % this removes page numbers from the frontmatter
  \begin{acknowledgements}
    I want to thank a few people.
  \end{acknowledgements}
  \begin{preface}
    This is an example of a thesis setup to use the reed thesis document
    class (for LaTeX) and the R bookdown package, in general.
  \end{preface}
  \hypersetup{linkcolor=black}
  \setcounter{tocdepth}{2}
  \tableofcontents

  \listoftables

  \listoffigures
  \begin{abstract}
    The preface pretty much says it all.
    
    \par
    
    Second paragraph of abstract starts here.
  \end{abstract}
  \begin{dedication}
    You can have a dedication here if you wish.
  \end{dedication}
\mainmatter % here the regular arabic numbering starts
\pagestyle{fancyplain} % turns page numbering back on

\hypertarget{introduction}{%
\chapter*{Introduction}\label{introduction}}
\addcontentsline{toc}{chapter}{Introduction}

Welcome to the \emph{R Markdown} thesis template. This template is based
on (and in many places copied directly from) the Reed College LaTeX
template, but hopefully it will provide a nicer interface for those that
have never used TeX or LaTeX before. Using \emph{R Markdown} will also
allow you to easily keep track of your analyses in \textbf{R} chunks of
code, with the resulting plots and output included as well. The hope is
this \emph{R Markdown} template gets you in the habit of doing
reproducible research, which benefits you long-term as a researcher, but
also will greatly help anyone that is trying to reproduce or build onto
your results down the road.

Hopefully, you won't have much of a learning period to go through and
you will reap the benefits of a nicely formatted thesis. The use of
LaTeX in combination with \emph{Markdown} is more consistent than the
output of a word processor, much less prone to corruption or crashing,
and the resulting file is smaller than a Word file. While you may have
never had problems using Word in the past, your thesis is likely going
to be about twice as large and complex as anything you've written
before, taxing Word's capabilities. After working with \emph{Markdown}
and \textbf{R} together for a few weeks, we are confident this will be
your reporting style of choice going forward.

\textbf{Why use it?}

\emph{R Markdown} creates a simple and straightforward way to interface
with the beauty of LaTeX. Packages have been written in \textbf{R} to
work directly with LaTeX to produce nicely formatting tables and
paragraphs. In addition to creating a user friendly interface to LaTeX,
\emph{R Markdown} also allows you to read in your data, to analyze it
and to visualize it using \textbf{R} functions, and also to provide the
documentation and commentary on the results of your project. Further, it
allows for \textbf{R} results to be passed inline to the commentary of
your results. You'll see more on this later.

\textbf{Who should use it?}

Anyone who needs to use data analysis, math, tables, a lot of figures,
complex cross-references, or who just cares about the final appearance
of their document should use \emph{R Markdown}. Of particular use should
be anyone in the sciences, but the user-friendly nature of
\emph{Markdown} and its ability to keep track of and easily include
figures, automatically generate a table of contents, index, references,
table of figures, etc. should make it of great benefit to nearly anyone
writing a thesis project.

\hypertarget{rmd-basics}{%
\chapter{Planteamiento de la necesidad}\label{rmd-basics}}

Intro

\hypertarget{plantamiento-del-problema}{%
\section{Plantamiento del problema}\label{plantamiento-del-problema}}

Tradicionalmente, las empresas contaban con un centro de datos donde
alojaban desde un par de servidores a cientos de estos. Un centro de
datos es un local o edificio donde reside el equipo de procesamiento de
una o varias empresas. El centro de datos debe constar de al menos una
habitación separada con suministro de energía independiente y control
clima Corp (2011). El consumo de energía eléctrica dentro de un centro
de datos es de alrededor de 45\%, por el equipo de tecnologías de la
información, en otras palabras, servidores, almacenamiento, y
telecomunicaciones. El otro 55\% de electricidad es consumida por las
instalaciones, donde se incluye sistema de distribución, fuentes de
alimentación ininterrumpidas, sistemas de enfriamiento Geng (2014).
Asimismo, mantener un centro de datos no solo requiere constante
monitoreo del equipo de computo, sino además, es necesario monitorizar
todos los componentes del centro de datos, tales como:
\begin{itemize}
\tightlist
\item
  Equipo de enfriamiento
\item
  Fuentes de poder
\item
  Seguridad
\item
  Crecimiento del centro de datos.
\end{itemize}
Por otro lado, el equipo de computo es una parte crucial para el negocio
de una empresa. Por esta razón, es necesario tener un plan contingencia
si algo llegara a fallar. Sí, alguna parte física de un fallase, y es
necesario reemplazarla, todo dependerá de que tan crítico sean los
procesos que se estén realizando en el equipo; ya que en algunas
ocasiones es necesario crear una ventana de mantenimiento, y en algunas
ocasiones, crear una ventana de mantenimiento puede llegar a tardar días
a meses. Por otro lado, es necesario tener en cuenta que existen
periodos de tiempo donde no se esta utilizando al máximo el equipo de
computo. Lo cual es un sobre aprovisionamiento de recursos de computo.
Un estudio realizado en 2012 por \emph{Natural Resources Defense
Council}, en el cual se estima que en promedio servidores que ejecutan
una sola aplicación tienden a utilizar entre 5\% y 15\% de poder de
computo Whitney \& Kennedy (2012). Por otra parte, se estima que en
grandes centros de datos el porcentaje de servidores obsoletos se
encuentra entre 20\% y 30\% Delforge (2014).

Por otro parte, si una empresa está teniendo un crecimiento bastante
acelerado donde la capacidad de computo con la que cuentan actualmente
será insuficiente en poco tiempo. La empresa se verá en la necesidad de
adquirir nuevo equipo de cómputo. Una vez que se realice la compra, el
tiempo de entrega depende del lugar de entrega y disponibilidad del
equipo en cuestión Dell (2015). Lo cual, puede generar problemas al no
tener la infraestructura necesaria para escalar el servicio, lo que
conlleva a una posibilidad de pérdida de clientes e inestabilidad de los
servicios.

\hypertarget{objetivo-general}{%
\section{Objetivo General}\label{objetivo-general}}

Crear un esquema para migración de infraestructura física de TI a la
nube, con el fin de agilizar y reducir el proceso de migración para el
usuario.

\hypertarget{objetivos-especificos}{%
\subsection{Objetivos específicos}\label{objetivos-especificos}}
\begin{itemize}
\item
  Crear un modelo del dominio del proceso de migración utilizando diseño
  basado en dominio (\emph{Domain Driven Design})
\item
  Desarrollar un esquema en base al modelo conceptual
\item
  Utilizar tecnologías de orquestación, microservicios y contenedores
  para la implementación del esquema.
\item
  Desarrollar una aplicación la cual implementará el esquema de
  migración y utilizará las tecnologías antes mencionadas.
\end{itemize}
\hypertarget{justificacion}{%
\section{Justificación}\label{justificacion}}

El computo en la nube ha transformado como las tecnologías de
información ofrecen sus servicios y, además, como son consumidas
Pinkelman (1993). Las compañías que arriendan su infraestructura o
servicios de computo son conocidos como proveedores de nube.
Actualmente, existen múltiples proveedores de nube alrededor del mundo.
Esto les da la oportunidad a las empresas de adquirir los servicios de
los proveedores, lo cual podría expandir sus mercados a uno global.
Estos servicios son presentados en un catálogo, en donde un usuario u
organización puede seleccionar el servicio deseado, utilizarlo y al
final solo pagar por el tiempo de consumo, como un pago más de
utilidades. Esto les da a los usuarios la posibilidad de acceder a un
gran poder de computo sin necesidad de una fuerte inversión inicial. Por
lo que brinda a las empresas grandes facilidades, ya que no solo reduce
el costo de inversión inicial. Además, permite administrar el equipo de
cómputo desde un solo lugar. Un ejemplo claro de esto es el caso del
Hospital Firmley Park donde se necesitaba reducir el tiempo
administrativo del equipo de cómputo, esto con el objetivo para poder
concentrarse en nuevas estrategias tecnológicas La solución a este
problema fue virtualizar las computadoras de escritorio de los
empleados, con el objetivo de poder administrar todo el equipo de
cómputo desde un solo lugar, y se accedía a estas utilizando conexión
remota (``Frimley,'' 2018).

Gracias al cómputo en la nube es posible tener múltiples respaldos de
aplicaciones o computadoras, no solo en diferentes computadoras, sino
hasta en diferentes países. Por esta razón, es posible brindar un mejor
rendimiento a los clientes, dado que es posible tener instancias de una
aplicación lo más cercano posible a él.

La migración de tecnologías de la información a la nube en muchas
ocasiones requiere de una gran planeación de la empresa, ya que existen
múltiples factores que se deben de considerar. Adicionalmente, si no se
tiene el conocimiento ni las herramientas para realizarlo puede resultar
en una gran problemática. Un ejemplo de esto es no tomar en
consideración los procesos que utilicen equipo \emph{legacy} para
algunas aplicaciones cruciales de la empresa Zou \& Kontogiannis (2000)
. El problema radica en que quizás la aplicación no interactúe
correctamente la aplicación con el servidor, y se tenga que adaptar de
alguna forma o modificar el código de esta. Lo cual puede generar un
retraso o fracaso en la adopción del cómputo en la nube.

Por último, se creará una aplicación que utilice un esquema de
detección, replicación y migración a la nube. Esta aplicación realizará
lo siguiente: Durante el periodo de detección, buscará el equipo de
cómputo dentro de una red, obtendrá información crucial de cada uno de
estos (CPU, RAM, disco duro). Esta información se utilizará para crear
máquinas virtuales. Durante el periodo de migración, en los equipos de
cómputo se iniciará la tarea de creación de imagines virtuales de estos.
Estas imágenes se utilizarán para subirlas a la nube.

\hypertarget{alcance-del-proyecto}{%
\section{Alcance del proyecto}\label{alcance-del-proyecto}}

\hypertarget{tipo-de-proyecto}{%
\section{Tipo de proyecto}\label{tipo-de-proyecto}}

\hypertarget{ref_labels}{%
\chapter{Marco Teórico}\label{ref_labels}}

A lo largo de esté capítulo se introducen tecnologías, arquitecturas de
software, las cuales son utilizadas el proyecto. Se da una breve
introducción sobre a la virtualización, la cual es una de las
principales características del cómputo en la nube. Adicionalmente, se
presenta se define el concepto de el cómputo en la nube, sus
características, modelos de despliegue y sus modelos de servicio. Se
introduce el \emph{DevOps} el cual

\hypertarget{virtualizacion}{%
\section{Virtualización}\label{virtualizacion}}

La virtualización es una forma de abstracción donde los componentes de
hardware son presentarlos de una forma lógica Kusnetzky (2011), con los
cuales, se pueden crear maquinas virtuales. En otras palabras, la
virtualización permite utilizar los recursos físicos de una computadora
para crear y alojar \emph{N} cantidad de máquinas virtuales. Para poder
realizar estas tareas es necesario un hipervisor. El hipervisor se
encarga de administrar los recursos de cómputo y proveerlos a las
máquinas virtuales. Existen dos tipos de hipervisores:
\begin{itemize}
\item
  Tipo 1 o \emph{Bare Metal}. Es conocido como nativo, ya que corre
  encima del hardware de la máquina, como si fuera un sistema operativo,
  esto permite un aislamiento verdadero entre cada sistema operativo de
  cada \emph{VM}.
\item
  Tipo 2 o \emph{Hosted}. Se ejecuta por encima del sistema operativo de
  la maquina anfitrión, esté se encarga de mostrar los recursos
  disponibles al hipervisor.
\end{itemize}
La ``máquina virtual'' fue desarrollada por \emph{IBM} en los años 60,
donde\,se tenía acceso concurrente e interactivo a una computadora
central desde varias terminales (monitores remotos). Cada máquina
virtual era una réplica representativa de la computadora central; es
decir, daba la impresión de estar físicamente en una computadora real
Ali \& Meghanathan (2011).~

\hypertarget{computo-en-la-nube}{%
\section{Cómputo en la nube}\label{computo-en-la-nube}}

El cómputo en la nube es un modelo para el aprovisionamiento de recursos
de cómputo, los cuales son presentados en forma de catálogo. En este
catálogo se presentan diversos tipos de servicios, tales como: redes,
servidores, almacenamiento, aplicaciones, que pueden ser rápidamente
aprovisionados y liberados con un esfuerzo mínimo de administración o de
interacción con el proveedor de servicios. Adicionalmente, el cómputo en
la nube está compuesto principalmente por cinco características
esenciales Mell, Grance, \& others (2011).~~

Las principales características del cómputo en la nube son:~ ~
\begin{itemize}
\item
  Servicio en demanda. El cliente puede adquirir el poder de cómputo
  necesario, automáticamente sin necesidad de interacción con el
  proveedor.~
\item
  Conectividad. Los servicios son disponibles a través de la red, y
  pueden ser accedidos a través de distintas plataformas.~~
\item
  Aprovisionamiento. El poder de cómputo\,está configurado para servir
  múltiples clientes en un modelo multi-cliente, con diferentes recursos
  físicos y virtuales que son asignados y reasignados de acuerdo con la
  demanda.~
\item
  Elasticidad. El aprovisionamiento puede ser elástico y escalable. En
  muchos casos puede ser automática, que escale rápidamente ya sea
  creciendo o disminuyendo dependiendo de la demanda.~
\item
  Servicios medidos. Los recursos del cómputo en la nube son controlados
  y optimizados utilizando una capacidad métrica, que se puede medir por
  almacenamiento, procesamiento, ancho de banda o usuarios activos. La
  utilización de recursos puede ser monitorizada, controlada y
  reportada. Proveyendo transparencia tanto para el proveedor como el
  consumidor de sus servicios.~~ ~
\end{itemize}
Adicionalmente existen múltiples formas de despliegue, las cuales pueden
tener diversos usos dependiendo en el ambiente en el que se utiliza. Los
modelos de despliegue son:~
\begin{itemize}
\item
  Nube privada. La infraestructura le pertenece a una empresa. Además,
  la nube puede estar administrada por la empresa dueña o por terceros.
  A su vez, puede residir dentro o fuera de la empresa.~
\item
  Nube comunitaria. Está constituida por múltiples comunidades u
  organizaciones las cuales comparten recursos para un fin en común.~
\item
  Nube pública. En este modelo la infraestructura de la nube puede ser
  provista por organizaciones o empresas para el uso del público y está
  la pertenece a proveedores de nube.~
\item
  Nube híbrida. Este modelo está compuesto de dos o más modelos.~
\end{itemize}
Uno de los grandes beneficios del modelo de nube privada, es que da, la
posibilidad de brindar una buena calidad de servicios, tiempo de
respuesta. Para poder implementar este modelo es necesario utilizar un
sistema operativo o plataforma, como los son: \emph{Openstack},
\emph{Cloudstack} Foundation (2017), \emph{VMware vCloud} (``Vcloud
suite,'' 2018).~\emph{Openstack} es un sistema operativo que controla
una gran cantidad de recursos de cómputo, almacenamiento y red Openstack
(2015). Los recursos son administrados por medio de un panel de control,
el cual, puede ser accedido por un navegador web, donde solo tienen
acceso los administradores y los. Por último, \emph{Openstack} es una
plataforma de software libre, la cual es la más utilizada por la
comunidad de software libre, además, está apoyada por múltiples
empresas, entre ellas, \emph{Red Hat}, \emph{HP},\emph{Google}, etc.~

El computo en la nube brinda diferentes niveles de servicios, en donde
dependiendo del nivel seleccionado se tendrá más capacidad de
personalización. Los principales niveles de servicios IBM (2018) :
\begin{itemize}
\item
  Software as a Service (SaaS). Las aplicaciones basadas en la nube, las
  cuales son ejecutadas en computadoras distantes ``en la nube'' que
  pertenecen y son operadas por otros, las cuales conectan las
  computadoras de los usuarios vía internet y, usualmente, navegador
  web.
\item
  Platform as a service (PaaS). Provee un ambiente donde todo lo que se
  requiere para soportar un ciclo de desarrollo e implementación de
  aplicaciones web, sin la necesidad de comprar o manejar hardware,
  software.
\item
  Infrastructure as a service (IaaS). Provee a las compañías con
  recursos computacionales incluyendo servidores, redes, almacenamiento,
  y espacio dentro de un centro de datos con pago por uso.
\end{itemize}
Los niveles de servicios que ofrece la nube han incrementado la
complejidad de los sistemas actuales, lo cual ha aumentado el número de
actividades de los administradores de sistemas. Estas actividades suelen
ser repetitivas como: crear máquinas virtuales, instalación de
actualizaciones, software o dependencias de este, aunque muchas de estas
actividades se pueden ser automatizadas utilizando \emph{scripts}. El
problema recae en que muchos \emph{scripts} son creados para realizar
tareas en un ambiente especifico, además, de que en muchas ocasiones no
son documentados apropiadamente, y en algunas ocasiones es necesario
tener un cierto nivel de conocimiento en programación, por otro lado,
para ejecutar \emph{scripts}, es necesario obtener acceso al servidor
donde serán ejecutarlos. Como resultado, ha sido necesario encontrar
nuevas formas de realizar estas tareas repetitivas.

\hypertarget{openstack}{%
\subsection{Openstack}\label{openstack}}

\hypertarget{devops}{%
\section{DevOps}\label{devops}}

En el mercado actual de las tecnologías de la información se ha
incrementado la velocidad de desarrollo y mantenimiento. La unión de las
áreas de desarrollo de software y operaciones ha creado una nueva área
llamada DevOps Geerling (2015). DevOps utiliza prácticas del desarrollo
de software en la administración de infraestructura como código
(\emph{Infrastructure-as-code} IaC). \emph{IaC} es un algoritmo que se
encarga de instalar dependencias, controladores necesarios por un
programa en específico en un servidor Wittig \& Wittig (2016). Por
último, DevOps promueve el uso de conjuntos de \emph{scripts}, modelos
para automatización y configuración. Esto con el propósito de reutilizar
código y mejorar los tiempos de desarrollo e implementación. Existen
múltiples herramientas para DevOps tales como:~

• \emph{Puppet} (2017a)~

• \emph{Chef} (2018a)~

• \emph{Ansible} (2017b)

La funcionalidad de estos programas es la administración y orquestación
de infraestructura.

\hypertarget{ansible}{%
\subsection{Ansible}\label{ansible}}

\emph{Ansible} solo necesita un nodo de administración, el cual cuenta
con un inventario y múltiples \emph{playbook}, estos últimos son los
archivos de configuración, implementación y orquestación. Dentro del
inventario se crean grupos en, los cuales se agregan los nombres de los
servidores o direcciones IPs. Dentro de \emph{playbook} se configuran
las tareas a realizar; estos archivos son creados utilizando el formato
\emph{YAML}, por otro lado, los módulos utilizan \emph{JSON}. Se utiliza
\emph{SSH} para la conectividad remota y no requiere abrir puertos
extras.~Adicionalmente, solo requiere tener instalado* \emph{Python}.

\hypertarget{arquitectura-monolitica}{%
\section{Arquitectura Monolítica}\label{arquitectura-monolitica}}

Tradicionalmente, las aplicaciones son programadas en una sola instancia
donde todas las actividades residen en una misma aplicación, también
conocido como arquitectura monolítica. Lo cual, genera una problemática
al tratar de actualizar el código de la aplicación, ya que muchas
ocasiones puede contar con cientos si no miles de líneas de código.
Además, son difíciles de comprender y mantener. Por otro lado, cuando se
requiere escalar una porción de la aplicación es necesario escalar toda
la aplicación, lo cual genera un mayor costo. La figura \ref{figura1} se
puede ver varios grupos de desarrolladores trabajando en la misma
aplicación sin tener definido que partes de la aplicación le pertenecen.
\begin{figure}[h!]
  \centering
  \includegraphics[scale=0.5]{./figure/Cap3/monoFig1.png}
  \caption{Aplicación monolítica}\label{figura1}
\end{figure}
A diferencia de la arquitectura monolítica, la arquitectura orientada a
servicios, esta constituida por múltiples servicios, los cuales trabajan
en conjunto para realizar una tarea. El desarrollo orientado a dominio
tiene como objetivo desarrollar una aplicación, la cual debe expresar el
objetivo de un negocio Adicionalmente, las características. La entrega
continua, virtualización, bajo demanda, automatización de
infraestructura, sistemas en escala. Estas son las características que
ayudan a implementar microservicios Newman (2015). Microservicios es un
conjunto de servicios autónomos que trabajan para alcanzar una meta en
común. Como se habló con anterioridad las aplicaciones monolíticas
limitan la forma de actualizar las aplicaciones ya que se requiere un
cierto periodo de tiempo para realizar mantenimiento en el cual se
realizan actualizaciones, además, de que limita las actualizaciones de
esquemas y formas de manejo de datos. El alcance de cada servicio se
enfoca en los alcances del negocio, esto permite reconocer con mayor
facilidad el alcance o dominio de cada servicio. Uno de los beneficios
que ofrece esta arquitectura es poder utilizar diversas tecnologías ya
sea lenguajes de programación, base de datos ya que cada servicio es
independiente de otro. Las características claves de los microservicios
son:
\begin{itemize}
\tightlist
\item
  Diseño orientado a dómino. Es un enfoque para el desarrollo de
  software con necesidades complejas mediante una profunda conexión
  entre la implementación y los conceptos del modelo y núcleo del
  negocio.
\item
  Principio de responsabilidad simple. De acuerdo con la filosofía de
  Unix cada servicio es responsable de una parte única de la
  funcionalidad y lo hace bien Raymond (2003).
\item
  Interfaz explicita
\item
  \emph{DURS independiente (Deploy, Update, Replace, Scale)} Cada
  servicio se puede implementar, actualizar, reemplazar y escalar de
  forma independiente.
\item
  \emph{Endpoints/ pipes} Cada microservicio posee su lógica de dominio
  y se comunica con otros a través de protocolos simples, como REST, el
  cual provee conectividad.
\end{itemize}
La forma en que se comunican entre sí los servicios es utilizando
llamadas a través de la red. Esto puede ser utilizando RPC (Remote
Procedure Calls) o REST (REpresentation State Transfer).

Un microservicio puede ser implementado en múltiples ambientes tales
como:
\begin{itemize}
\tightlist
\item
  Máquinas virtuales
\item
  Contenedores
\item
  Plataforma como servicio (PaaS)
\end{itemize}
Múltiples microservicios pueden ser implementados en el mismo ambiente,
aunque no es recomendado, ya que reduce los puntos de falla. En la
figura 4 se muestra un ejemplo de una aplicación utilizando la
arquitectura de microservicios, en donde se tiene bien definido los
dominios de cada servicio. Como se muestra en la \ref{figura2}, se
tienen multiples servicios en dominios bien definidos, ademas, de tener
bien definido como se comunican entre ellos.
\begin{figure}[h!]
  \centering
  \includegraphics[scale=0.5]{./figure/Cap3/monoFig2.png}
  \caption{Microservicios}\label{figura2}
\end{figure}
\hypertarget{diseno-basado-en-dominios}{%
\section{Diseño Basado en Dominios}\label{diseno-basado-en-dominios}}

Los fundamentos principales de DDD están basados en la discusión,
escuchar, entendimiento, descubrimiento, y valores de negocio, todo esto
para poder centralizar el conocimiento. Si eres capaz de entender el
negocio en el cual se basará, por lo menos podrá participar en el
modelado del software y podrá participar en el proceso de crear el
lenguaje ubicuo. (ubicuo. Está presente a un mismo tiempo en todas
partes)

Durante este proceso de creado del lenguaje ubicuo es necesario entablar
conversaciones con expertos del dominio. Los expertos del dominio son
aquellos que conocen como funciona el negocio. Un experto del domino no
esta basado en ``títulos''. Ya que, existen personas que conocen su área
de negocio bastante bien. Por lo tanto, ellos pueden proveer información
vital para el lenguaje ubicuo.

Es un modelo basado en software, el cual esta basado en un dominio de
negocio. También considerado como modelo objeto, donde existen objetos,
los cuales tienen datos y comportamientos en base al negocio. Crear un
modelo del domino es esencial para poder utilizar DDD. Utilizando DDD
los modelos del dominio tienden a ser pequeños y enfocados.

Permitir que los expertos del dominio y desarrolladores trabajen en
conjunto, lo cual producirá un software que este basado en el negocio.
Centralizar el conocimiento es clave, porque con esto la empresa es
capaz de garantizar la comprensión del software. The design is the code,
and the code is the design.

DDD provee técnicas de desarrollo de software, las cuales se encargan
del diseño estratégico y táctico. Diseño estratégico ayuda a entender
cuales son las inversiones que se tiene realizar con el software, que
tipo de software existe para poder obtener un software rápido y seguro.
Diseño táctico ayuda a desarrollar un solo modelo de la solución.

\hypertarget{aspectos-principales-de-ddd}{%
\subsection{Aspectos principales de
DDD}\label{aspectos-principales-de-ddd}}
\begin{itemize}
\item
  Acerca a los expertos del dominio y desarrolladores para trabajar en
  conjunto para reflejar el modelo mental del experto. Al trabajar
  juntos los expertos del dominio y desarrolladores su principal
  objetivo es crear un lenguaje ubicuo. Este lenguaje permitirá tener
  una mejor comunicación y un mayor entendimiento sobre el dominio del
  negocio.
\item
  DDD aborda las iniciativas estratégicas de la empresa. Aunque DDD
  incluya técnicas de análisis, esta mas enfocado con la estrategia de
  dirección de la empresa. Los aspectos técnicos de la estrategia del
  diseño tiene como objetivo crear bounding system y preocupaciones de
  negocios.
\item
  Tácticas de diseño permiten a los desarrolladores producir un software
  que esta correctamente codificado en base a los conocimientos de los
  expertos del dominio, es escalable, y permite cómputo distribuido.
\end{itemize}
\hypertarget{valores-y-beneficios-de-ddd}{%
\subsection{Valores y beneficios de
DDD}\label{valores-y-beneficios-de-ddd}}
\begin{enumerate}
\def\labelenumi{\arabic{enumi}.}
\item
  Organizaciones ganan un modelo útil de su dominio. El objetivo de DDD
  es invertir todos los esfuerzos en lo que importa más del negocio. Se
  enfoca en el dominio central (Core domain). Otros modelos existen para
  dar soporte al dominio central.
\item
  Una definición refinada y precisa del negocio es desarrollado. Cuando
  el modelo es refinado una y otra vez, se desarrolla un mejor
  entendimiento el cual se puede utilizar como herramienta de análisis.
\item
  Expertos del dominio contribuyen al diseño del software. Cuando los
  expertos comparten sus conocimientos entre ellos, permite crear un
  mejor entendimiento del negocio. Esto, ayuda a el crecimiento de la
  empresa. Adicionalmente, los desarrolladores comparten un lenguaje
  ubicuo con los expertos.
\item
  Mejor experiencia de usuario. Usualmente, la retroalimentación del
  usuario puede transformarse en un mejor reflejo del modelo del
  dominio. Cuando el software deja mucho al entendimiento del usuario,
  los usuarios necesitan ser entrenados para poder utilizarlo. En
  esencia, el usuario solo transfiere su entendimiento del software a
  datos, los cuales son introducidos al software. Estos datos son
  guardados. Sí el usuario no entiende exactamente que es lo que
  necesita introducir, entonces, los resultados no son los correctos.
\item
  Los límites son claros, los cuales son planteados alrededor de los
  modelos. Los desarrolladores son orientados a utilizar un enfoque de
  negocio.
\item
  La arquitectura empresarial es mejor organizada. Cuando los límites
  del contexto son bien definidos y cuidadosamente particionados, todos
  los equipos tienen un claro entendimiento de donde y porqué las
  integraciones son necesarias. Los límites son explícitos, y las
  relaciones entre ellos también.
\item
  Ágil, iterativo(repetitivo), modelado continuo es usado. El objetivo
  de DDD es refinar el modelo mental de los expertos del dominio a un
  modelo útil para el negocio.
\item
  Nuevas herramientas, estratégica y tácticas, son utilizadas. El límite
  contextual da al equipo límites de modelado en donde se crea una
  solución para un problema especifico en el dominio. Dentro de un
  límite contextual un lenguaje ubicuo es creado. Este, es utilizado por
  el equipo y en el modelo del software. Dentro de un límite de modelado
  pueden utilizar tácticas: Aggregates, Entidades, Objeto Valor,
  Servicios, Eventos del Dominio, entre otros.
\end{enumerate}
\hypertarget{materiales-y-metodos}{%
\chapter{Materiales y métodos}\label{materiales-y-metodos}}

Intro de Cap

\hypertarget{descripcion-del-area-de-estudio}{%
\section{Descripción del área de
estudio}\label{descripcion-del-area-de-estudio}}

\hypertarget{materiales}{%
\section{Materiales}\label{materiales}}

\hypertarget{hardware}{%
\subsection{Hardware}\label{hardware}}

Las herramientas utilizadas fueron las siguientes:
\begin{enumerate}
\def\labelenumi{\arabic{enumi}.}
\tightlist
\item
  Computadora Workstation que cuenta con:
\end{enumerate}
\begin{itemize}
\tightlist
\item
  Procesador \emph{Intel} i7 6800k @4.2GHz de 6 núcleos
\item
  Memoria RAM total de 32 GB @ 2666MHz
\item
  Discos duros:
  \begin{itemize}
  \tightlist
  \item
    \emph{Samsung} 960 EVO 500 GB NVMe SSD
  \item
    SSD 250 GB
  \item
    HDD 1 TB 7200 RPM
  \end{itemize}
\end{itemize}
\begin{enumerate}
\def\labelenumi{\arabic{enumi}.}
\setcounter{enumi}{1}
\tightlist
\item
  Servidor Dell PowerEdge 2950
\end{enumerate}
\begin{itemize}
\tightlist
\item
  2x procesadores Xeon Dual Core
\item
  Memoria RAM total 16 GB
\item
  Discos duros de 200 GB 1500 RPM
\end{itemize}
\begin{enumerate}
\def\labelenumi{\arabic{enumi}.}
\setcounter{enumi}{2}
\tightlist
\item
  Notebook Gateway nv52
\end{enumerate}
\begin{itemize}
\tightlist
\item
  Disco duro de 320GB.
\item
  Procesador AMD Athlon X2 Dual-Core QL-64 2.10Ghz.
\item
  Memoria RAM 4 GB
\end{itemize}
\begin{enumerate}
\def\labelenumi{\arabic{enumi}.}
\setcounter{enumi}{3}
\tightlist
\item
  Notebook ASUS Zenbook
\end{enumerate}
\begin{itemize}
\tightlist
\item
  Procesador \emph{Intel} i5-5200U @ 2.2Ghz
\item
  memoria RAM 8 GB
\item
  Disco duro SSD 250 GB
\end{itemize}
\hypertarget{software}{%
\subsection{Software}\label{software}}

Sistemas operativos utilizados:
\begin{itemize}
\tightlist
\item
  Fedora 26 Workstation
\item
  Fedora 27 Workstation
\item
  Windows 7
\end{itemize}
En cuanto a software se utilizó:
\begin{itemize}
\tightlist
\item
  Python 3
\item
  Flask 0.12.2
\item
  Ansible 2.5.2
\item
  Kubernetes 1.9.8
\item
  Devstack
\item
  Docker 18.03.1-ce, build 9ee9f40
\item
  VIM 8.0.1806
\item
  RStudio 1.1.442
\item
  Kile 4.14.32
\end{itemize}
\hypertarget{metodos}{%
\section{Métodos}\label{metodos}}

\hypertarget{devstack}{%
\subsection{Devstack}\label{devstack}}

\emph{Devstack} Es una serie de scripts y utilidades para poder
desplegar una nube \emph{Openstack} (2018b). El proceso de instalación
se muestra en el apéndice A.

\hypertarget{configuracion}{%
\subsubsection{Configuración}\label{configuracion}}

Tanto la configuración de usuarios y sus contraseñas son las de fabrica.

\hypertarget{docker}{%
\subsection{Docker}\label{docker}}

\emph{Docker} es una plataforma para la creacion, despliegue de
contenedores Mouat (2015).

\hypertarget{modelacion-utilizando-diseno-basado-en-dominios}{%
\chapter{Modelacion utilizando diseño basado en
dominios}\label{modelacion-utilizando-diseno-basado-en-dominios}}

Utilizando el esquema de migración. El usuario debe contar con una
cuenta en Openstack, en este proyecto se utilizaron los usuarios y
contraseñas de fabrica. Se proveerá un \emph{script}, el cual obtendrá
información básica para crear una maquina virtual. Con esta información
se crea un inventario donde el usuario puede seleccionar, eliminar
equipo de computo. 0

\hypertarget{requisitos}{%
\section{Requisitos}\label{requisitos}}
\begin{itemize}
\tightlist
\item
  Permita reconocer el equipo de computo en la red.
\item
  Crear un inventario con el equipo de computo del usuario.
  \begin{itemize}
  \tightlist
  \item
    Permita seleccionar cuales serán las computadoras para la migración.
  \end{itemize}
\item
  Facilitar la creación de respaldos de disco duro
\item
  Migración de los respaldos a la nube
\end{itemize}
Las computadoras del usuario deberán tener lo siguiente:
\begin{itemize}
\tightlist
\item
  \emph{Windows} 7, 8.1 o 10
\item
  \emph{Powershell} 3.0 en adelante
\item
  \emph{NET} 4.0
\item
  \emph{WinRM} deberá ser creado y activado.
\end{itemize}
El modelo siempre debe ser construido teniendo en cuenta las
consideraciones de diseño y software. Esto, con el propósito de diseñar
un modelo, el cual pueda ser expresado apropiadamente en software.

Una vez analizado los requisitos se encontraron tres subdominios:
\begin{itemize}
\tightlist
\item
  Inventario
\item
  Migración
\item
  Orquestación Vernon (2013)
\end{itemize}
figura pagina 55 de implementacion
\begin{figure}[h!]
  \centering
  \includegraphics[scale=0.5]{./figure/Cap4/plantillaDDD.png}
  \caption{Modelado del dominio}\label{DDDplantilla}
\end{figure}
Cada uno de estos se explicará a continuación.

\hypertarget{subdominio-inventario}{%
\subsection{Subdominio Inventario}\label{subdominio-inventario}}

\hypertarget{conclusion}{%
\chapter*{Conclusion}\label{conclusion}}
\addcontentsline{toc}{chapter}{Conclusion}

If we don't want Conclusion to have a chapter number next to it, we can
add the \texttt{\{-\}} attribute.

\textbf{More info}

And here's some other random info: the first paragraph after a chapter
title or section head \emph{shouldn't be} indented, because indents are
to tell the reader that you're starting a new paragraph. Since that's
obvious after a chapter or section title, proper typesetting doesn't add
an indent there.

\appendix

\hypertarget{the-first-appendix}{%
\chapter{The First Appendix}\label{the-first-appendix}}

This first appendix includes all of the R chunks of code that were
hidden throughout the document (using the \texttt{include\ =\ FALSE}
chunk tag) to help with readibility and/or setup.

\textbf{In the main Rmd file}
\begin{Shaded}
\begin{Highlighting}[]
\CommentTok{# This chunk ensures that the thesisdown package is}
\CommentTok{# installed and loaded. This thesisdown package includes}
\CommentTok{# the template files for the thesis.}
\ControlFlowTok{if}\NormalTok{(}\OperatorTok{!}\KeywordTok{require}\NormalTok{(devtools))}
  \KeywordTok{install.packages}\NormalTok{(}\StringTok{"devtools"}\NormalTok{, }\DataTypeTok{repos =} \StringTok{"http://cran.rstudio.com"}\NormalTok{)}
\ControlFlowTok{if}\NormalTok{(}\OperatorTok{!}\KeywordTok{require}\NormalTok{(thesisdown))}
\NormalTok{  devtools}\OperatorTok{::}\KeywordTok{install_github}\NormalTok{(}\StringTok{"ismayc/thesisdown"}\NormalTok{)}
\KeywordTok{library}\NormalTok{(thesisdown)}
\end{Highlighting}
\end{Shaded}
\textbf{In Chapter \ref{ref-labels}:}

\hypertarget{the-second-appendix-for-fun}{%
\chapter{The Second Appendix, for
Fun}\label{the-second-appendix-for-fun}}

\backmatter

\hypertarget{references}{%
\chapter*{References}\label{references}}
\addcontentsline{toc}{chapter}{References}

\markboth{References}{References}

\noindent

\setlength{\parindent}{-0.20in}
\setlength{\leftskip}{0.20in}
\setlength{\parskip}{8pt}

\hypertarget{refs}{}
\leavevmode\hypertarget{ref-Cap3_2_mT}{}%
Ali, I., \& Meghanathan, N. (2011). Virtual machines and
networks-installation, performance study, advantages and virtualization
options. \emph{arXiv Preprint arXiv:1105.0061}.

\leavevmode\hypertarget{ref-Cap1_4_PdP}{}%
Corp, R. (2011). \emph{R\&M data center handbook}. Reichle \& De-Massari
AG.

\leavevmode\hypertarget{ref-Cap1_2_PdP}{}%
Delforge, P. (2014). America's data centers are wasting huge amounts of
energy. WSP Environment \& Energy, LLC Natural Resources Defense
Council. Retrieved from
https://www.nrdc.org/sites/default/files/data-center-efficiency-assessment-IB.pdf.

\leavevmode\hypertarget{ref-Cap1_3_PdP}{}%
Dell. (2015). Shipping and delivery. Retrieved from
\url{http://www.dell.com/support/Contents/mx/en/mxdhs1/article/eSupport-Order-Support/shipping-and-delivery?~ck=mn}

\leavevmode\hypertarget{ref-Cap3_5_mT}{}%
Foundation, T. A. S. (2017). Apache cloudstack. Retrieved from
\url{https://cloudstack.apache.org/}

\leavevmode\hypertarget{ref-Cap1_Hospital}{}%
Frimley. (2018). \emph{Vmware.com}. Retrieved from
\url{https://www.vmware.com/content/dam/digitalmarketing/vmware/en/pdf/casestudy/customers/vmware-frimley-park-hospital-13q2-en-case-study.pdf}

\leavevmode\hypertarget{ref-Cap3_8mT}{}%
Geerling, J. (2015). \emph{Ansible for devops: Server and configuration
management for humans}. LeanPub.

\leavevmode\hypertarget{ref-Cap1_5_PdP}{}%
Geng, H. (2014). \emph{Data center handbook}. John Wiley \& Sons.

\leavevmode\hypertarget{ref-ibmcloud}{}%
IBM. (2018). Learn what is cloud computing? Retrieved from
\url{https://www.ibm.com/cloud/learn/what-is-cloud-computing}

\leavevmode\hypertarget{ref-Cap3_1_mT}{}%
Kusnetzky, D. (2011). \emph{Virtualization: A manager's guide}. "
O'Reilly Media, Inc.".

\leavevmode\hypertarget{ref-Cap3_3_mT}{}%
Mell, P., Grance, T., \& others. (2011). The nist definition of cloud
computing.

\leavevmode\hypertarget{ref-Cap4_Docker}{}%
Mouat, A. (2015). \emph{Using docker: Developing and deploying software
with containers}. " O'Reilly Media, Inc.".

\leavevmode\hypertarget{ref-Cap3_Microservicios}{}%
Newman, S. (2015). \emph{Building microservices: Designing fine-grained
systems}. " O'Reilly Media, Inc.".

\leavevmode\hypertarget{ref-Cap3_4_mT}{}%
Openstack. (2015). Software. Retrieved from
\url{ttps://www.openstack.org/software/}

\leavevmode\hypertarget{ref-Cap1_ACMInroads}{}%
Pinkelman, J. (1993). Computing changes: An industry perspective.
\emph{ACM Inroads}, \emph{4}(4), 39--42.

\leavevmode\hypertarget{ref-Cap3_ArtOfLinux}{}%
Raymond, E. S. (2003). \emph{The art of unix programming}.
Addison-Wesley Professional.

\leavevmode\hypertarget{ref-Cap3_6_mT}{}%
Vcloud suite. (2018). \emph{VMWare}. Retrieved from
\url{https://www.vmware.com/products/vcloud-suite.html}

\leavevmode\hypertarget{ref-Cap4_ImplementingPlantillaDibujo}{}%
Vernon, V. (2013). \emph{Implementing domain-driven design} (p. 55).
Addison-Wesley.

\leavevmode\hypertarget{ref-Cap1_1_PdP}{}%
Whitney, J., \& Kennedy, J. (2012). The carbon emissions of server
computing for small-to medium-sized organization. WSP Environment \&
Energy, LLC Natural Resources Defense Council. Retrieved from
http://www. wspenvironmental.
com/media/docs/ourlocations/usa/NRDC-WSP\_Cloud\_Computing. pdf.

\leavevmode\hypertarget{ref-Cap3_7mT}{}%
Wittig, M., \& Wittig, A. (2016). \emph{Amazon web services in action}.
Manning.

\leavevmode\hypertarget{ref-Cap1_WebBased}{}%
Zou, Y., \& Kontogiannis, K. (2000). Web-based specification and
integration of legacy services. In \emph{Proceedings of the 2000
conference of the centre for advanced studies on collaborative research}
(p. 17). IBM Press.

\leavevmode\hypertarget{ref-Cap3_Puppet}{}%
(2017a). \emph{Puppetlabs}. Retrieved from
\url{https://www.puppetlabs.com}

\leavevmode\hypertarget{ref-Cap3_ansible}{}%
(2017b). \emph{Red Hat Ansible DevOps made simple}. Retrieved from
\url{https://www.ansible.com}

\leavevmode\hypertarget{ref-Cap3_9mT}{}%
(2018a). \emph{Chef}. Retrieved from \url{https://www.chef.io/chef/}

\leavevmode\hypertarget{ref-Cap4_devstack}{}%
(2018b). \emph{Devstack}. Retrieved from
\url{https://docs.openstack.org/devstack/latest/}


% Index?

\end{document}
